%! TeX program = lualatex
%! TEX options = -synctex=1 -interaction=nonstopmode -file-line-error --shell-escape "%DOC%"
\documentclass[a4paper,11pt]{article}
\usepackage{polski}
% \usepackage{libertine}
\usepackage[top=1.5cm, bottom=1.5cm, left=2.5cm, right=2.5cm]{geometry}
\usepackage{url}
\usepackage{graphicx}
\usepackage{amsfonts}
\usepackage{amsmath}
\usepackage{enumitem}
\usepackage{multicol}
\usepackage{fontspec}
\usepackage{bm}

\newfontfamily\DejaSans{DejaVu Sans}

\everymath{\displaystyle}
\newcommand{\dm}[1]{\displaystyle{#1}}
\newcommand{\RR}{\mathbb{R}}

\setlength\parindent{0.5pt} 

\begin{document}

\begin{center}
  {\large\textbf{Lista 9}}
\end{center}

\hrulefill
\begin{center}
    \textbf{Zadania do deklaracji (poniedziałek)}
\end{center}

\bigskip

\textbf{Zadanie 1} Wyznaczyć równanie płaszczyzny przechodzącej przez
punkt $(1,1,3)$ i stycznej do powierzchni o równaniu $z = 2x^2 + y^2$.

\bigskip

\textbf{Zadanie 2} Oblicz pochodną kierunkową funkcji $f$ w punkcie
$a=(1,0,1,0)$ w kierunku $v = (0, \frac{1}{\sqrt{2}}, 0,
\frac{1}{\sqrt{2}})$ dla funckji $f: \mathbb{R}^4 \to \mathbb{R}$ $f(x) =
\exp(x_1 x_2 x_3 x_4)$. Wskazówka: nie trzeba korzystać z definicji.

\bigskip

\textbf{Zadanie 3} Niech $f : \mathbb{R}^2 \to \mathbb{R}$ będzie
różniczkowalna oraz $\forall_{x\in \mathbb{R}^2} \frac{\partial
f}{\partial x_2}(x) = 5 \frac{\partial f}{\partial x_1} (x)$. Znajdź
rówanie prostej, na której funkcja jest stała. Wskazówka: podobne zadanie
było na ostatnich ćwiczeniach.

\bigskip

\textbf{Zadanie 4} Niech $f(x,y)=(x^3-x-y)(2x-y-2)$ dla $x,y \in \mathbb
R$. Proszę wyznaczyć wszystkie punkty krytyczne funkcji $f$.


\hrulefill

\begin{center}
    \textbf{Zadania na zajęcia}
\end{center}

\bigskip

\textbf{Zadanie 5} Dla każdego  punktu krytycznego z poprzedniego zadania
rozpoznać, czy $f$ ma w nim lokalne ekstremum. 

\bigskip

\textbf{Zadanie 6} Niech $f:\mathbb R^3 \to \mathbb R$ będzie dana wzorem
$$
  f(x,y,z)=x^2+y^2+z^2+\kappa y z\, .
  $$
  Znaleźć wszystkie wartości $\kappa$, dla których $f$ ma lokalne minimum
  w $(0,0,0)$. 

\bigskip

\textbf{Zadanie 7} Niech $f(x,y)=x^3y-3x^2y+y^2$ dla $x,y \in \mathbb R$.
Proszę:
\begin{itemize}
    \item wyznaczyć wszystkie punkty krytyczne funkcji $f$,
    \item dla każdego z tych punktów rozpoznać, czy $f$ ma w nim lokalne
        ekstremum. 
\end{itemize}

\bigskip

\textbf{Zadanie 8} Poszukaj minimum i maksimum podanych funkcji:

\begin{enumerate}
    \item $z = (x-1)^2 + 2y^2$
    \item $z = (x-1)^2 - 2y^2 $
    \item $z = x^2 + xy + y^2 - 2x - y $
    \item $z = x^3y^2 (6-x-y) $ dla $x>0,y>0$
    \item $u = x^2 + y^2 + z^2 -xy + x - 2z$
\end{enumerate}

\bigskip

\textbf{Zadanie 9} Wyznacz wymiary takiego prostopadłościennego akwarium
bez „górnej przykrywki” o objętości 100 litrów, na którego zbudowanie
potrzeba zużyć najmniejszej powierzchni szyb.

\bigskip

\textbf{Zadanie 10} Wyznacz wymiary prostopadłościennego 100 litrowego
akwarium o szkielecie zbudowanym z prętów (wzdłuż wszystkich krawędzi),
na zbudowaniu którego potrzeba najmniejszej długości prętów.

\end{document}

