%! TeX program = lualatex
%! TEX options = -synctex=1 -interaction=nonstopmode -file-line-error --shell-escape "%DOC%"
\documentclass[a4paper,11pt]{article}
\usepackage{polski}
% \usepackage{libertine}
\usepackage[top=1.5cm, bottom=1.5cm, left=2.5cm, right=2.5cm]{geometry}
\usepackage{url}
\usepackage{graphicx}
\usepackage{amsfonts}
\usepackage{amsmath}
\usepackage{enumitem}
\usepackage{multicol}
\usepackage{fontspec}
\usepackage{bm}

\newfontfamily\DejaSans{DejaVu Sans}

\everymath{\displaystyle}
\newcommand{\dm}[1]{\displaystyle{#1}}
\newcommand{\RR}{\mathbb{R}}
\newcommand{\CC}{\mathbb{C}}

\setlength\parindent{0.5pt} 

\begin{document}

\begin{center}
  {\large\textbf{Lista 10}}
\end{center}

\hrulefill
\begin{center}
    \textbf{Zadania do deklaracji (piątek)}
\end{center}

\bigskip

\textbf{Zadanie 1-4} Znajdź ekstrema (jeśli istnieją) funkcji 

\begin{enumerate}
  \item $f: \mathbb{R}^3 \to \mathbb{R}:$ $f(x,y,z) = x+ \frac{y^2}{4x} +
    \frac{z^2}{y} + \frac{z}{2}$ dla $x,y,z > 0$.
  \item $f: \mathbb{R}^2 \to \mathbb{R}: f(x,y) = x^2 + 4xy$
  \item $f: \mathbb{R}^2 \to \mathbb{R}: f(x,y) = xy(1-x)(2-y)$
  \item $f: \mathbb{R}^3 \to \mathbb{R}: f(x,y) = y^2 + z^2 + 2xy$
\end{enumerate}

\bigskip

\hrulefill

\bigskip

\textbf{Zadanie 5} Wyznacz wartości parametru $a$ dla którego funkcja
$h(x,y) = ay(e^x - 1) + x\sin x + 1 - \cos y$ ma ekstremum lokalne w
punkcie $(0,0)$.

\bigskip

\textbf{Zadanie 6} Sprawdzić, że funkcja $f(x,y) = e^{-x} (x e^{-x} +
\cos y), x,y \in \mathbb{R}$ ma nieskończenie wiele wiele punktów
krytycznych, a w każdym z nich – maksimum lokalne właściwe

\bigskip

\textbf{Zadanie 7} Oblicz $\frac{\partial^2 f}{\partial x_2 \partial
x_1}(0)$ i $\frac{\partial^2 f}{\partial x_1 \partial x_2}(0)$ dla $f:
\mathbb{R}^2 \to \mathbb{R}$ zadanej wzorem 

\[
f(x) = 
\begin{cases}
  0 & \text{dla } x=0 \\
  x_1 x_2 \frac{x_1^2 - x_2^2}{\|x\|^2} & \text{dla } x \neq 0
\end{cases}
.\] 
Czy ta funkcja jest klasy $C^2$?

\bigskip

\textbf{Zadanie 8} Koryta dwóch rzek (na pewnym obszarze) można w
przybliżeniu opisać przez parabolę $y=x^2$ oraz prostą linię  $x-y-2=0$.
Potrzeba połączyć te dwie rzeki kanałem o najmniejszej długości. Przez
jakie punkty będzie przepływał ten kanał?

\bigskip

\textbf{Zadanie 9} Znajdź maksimum funkcji (a) $f(x,y) = x+y$ oraz (b)
$f(x,y) =(x+y)^2$ na okręgu $x^2+y^2 = 1$.

\bigskip

\textbf{Zadanie 10} Znajdź kresy funkcji $f$ zadanych poniższymi wzorami
na zbiorze $M$, zbadaj czy są one osiągane.

\begin{enumerate}
  \item $f(x,y) = x^2 + y^2 \qquad M=\{(x,y) \in \mathbb{R}^2 : 2x + 3y = 7\}$ 
  \item $f(x,y) = \sqrt{(x-2)^2 + y^2} \qquad M=\{(x,y) \in \mathbb{R}^2 :
    x^2 + y^2= 1\}$ 
  \item $f(x,y,z) = xyz \qquad M=\{(x,y,z) \in \mathbb{R}^3 : x^2 + y^2 +
      z^2= x+y+z = 1\}$ 
  \item $f(x,y) = Ax + By + C \qquad M=\{(x,y) \in \mathbb{R}^2 : x^2 +
    y^2 = 1\}$ 
  \item 
    $f(x,y)=\displaystyle\frac{x \ln(1+y) }{2x^2+y^2}, \qquad A=
    \{(x,y): 0<x\le y\le 1\}$.
\end{enumerate}

\bigskip

\textbf{Zadanie 11} Czy istnieje punkt z płaszczyzny w $\mathbb{R}^3$ o
równaniu $3x − 2z = 0$, dla którego suma kwadratów odległości od punktów
$(1, 1, 1)$ i $(2, 3, 4)$ jest najmniejsza?  Jeśli tak, to znajdź
wszystkie takie punkty.

\bigskip

\textbf{Zadanie 12} Załóżmy, że funkcja $f: \RR^2 \to \RR^2$ jest
różniczkowalna i $f(x,y) = (u(x,y), v(x,y))$, gdzie
$u,v: \RR^2 \to \RR$. Funkcję $f$ traktujemy jako odwzorowanie $f:
\CC \to \CC$, tzn. dla $z = x + iy$ (gdzie $i^2 = -1$)
$$ f(x) = u(z) + i v(z). $$
Udowodnić, że pochodna zespolona 
$$ f'(z) = \lim_{w \to 0} \frac{f(z+w) - f(z)}{w}, \qquad
(\text{gdzie } w \in \CC) $$
istnieje wtedy i tylko wtedy, gdy
$$ \frac{\partial u}{\partial x}(z) =
\frac{\partial v}{\partial y} (z) \quad \text{i}
\quad
\frac{\partial u}{\partial y}(z) =
-\frac{\partial v}{\partial x} (z).  $$
Wykazać, że funkcja $g: \CC \to
\CC$,  $g(z) = \overline z$, nie ma
pochodnej zespolonej.

\bigskip

\textbf{Zadanie 13} Niech $f:\RR^d \to \RR$ będzie funkcją
\textit{jednorodną stopnia $1$}, tzn. taką, że dla dowolnego $t\in \RR$ i
$x\in \RR^d$ zachodzi $f(t x)=t f(x)$. Wykazać, że $f$ jest
różniczkowalna w $0$ wtedy i tylko wtedy, gdy $f$ jest funkcją liniową. 

\end{document}

