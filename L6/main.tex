%! TeX program = lualatex
%! TEX options = -synctex=1 -interaction=nonstopmode -file-line-error --shell-escape "%DOC%"
\documentclass[a4paper,11pt]{article}
\usepackage{polski}
% \usepackage{libertine}
\usepackage[top=1.5cm, bottom=1.5cm, left=2.5cm, right=2.5cm]{geometry}
\usepackage{url}
\usepackage{graphicx}
\usepackage{amsfonts}
\usepackage{amsmath}
\usepackage{enumitem}
\usepackage{multicol}
\usepackage{fontspec}

\newfontfamily\DejaSans{DejaVu Sans}

\everymath{\displaystyle}

\setlength\parindent{0.5pt} 

\begin{document}

\begin{center}
  {\large\textbf{Lista 6}}
\end{center}

\hrulefill
\begin{center}
    \textbf{Zadania do deklaracji (poniedziałek)}
\end{center}

\bigskip

\textbf{Zadanie 1} Dla $f:[a,b] \to \mathbb{R}$ określamy:

\begin{itemize}
    \item \textit{długość wykresu} $f$, o ile $f$ jest klasy $C^1$, wzorem 
    $$\int_a^b \sqrt{1+(f'(x))^2} ~dx;$$
    \item \textit{pole powierzchni obrotowej} powstałej przez obrót wykresu $f$ 
    (zawartego w płaszczyźnie $XY$) wokół osi $X$ w przestrzeni $XYZ$, 
    o ile $f$ jest klasy $C^1$, wzorem
    $$2\pi \int_a^b f(x) \sqrt{1+(f'(x))^2}~dx;$$
    \item \textit{objętość bryły obrotowej ograniczonej powyższą powierzchnią 
    obrotową i płaszczyznami "$x=a$" oraz "$x=b$", o ile $f$ jest ciągła, wzorem}
    $$\pi \int_a^b (f(x))^2 ~dx.$$
\end{itemize}

W oparciu o powyższe wzory oblicz:
\begin{enumerate}
    \item  długość okręgu o promieniu $r$,
    \item objętość kuli o promieniu $r$,
    \item pole powierzchni sfery o promieniu $r$,
    \item  objętość walca obrotowego o promieniu podstawy $r$ i wysokości $h$,
    \item pole powierzchni bocznej walca obrotowego o promieniu podstawy $r$ i wysokości $h$,
    \item objętość stożka obrotowego o promieniu podstawy $r$ i wysokości $h$,
    \item pole powierzchni bocznej stożka obrotowego o promieniu podstawy $r$ i wysokości $h$.
\end{enumerate}

\textit{Proszę zrobić 4 wybrane przez was zadania z wymienionych powyżej.} 

\bigskip

\hrulefill

\textbf{Zadanie 2 (Róg Gabriela)} Oblicz pole powierzchni oraz objętość
bryły obrotowej ograniczonej przez powierzchnię powstałą w wyniku obrtu
wokół osi $OX$ wykresu funkcji $f(x) = \frac{1}{x}$ określonej na
przedziale $[1,\infty]$. Widzisz pewien paradoks?

\bigskip

\textbf{Zadanie 3} Obliczyć pole figury ograniczonej krzywymi:
\begin{itemize}
    \item  $y = x^2 - 6x + 10 \quad , \quad y = 6x - x^2$
    \item $y = \frac{1}{1+x^2} \text{~(czarownica Agnesi)} \quad , \quad 
        y = \frac{x^2}{2}$
    \item $y=e^x, y=e^{-x}, x =1$
    \item $^\ast$ o parametryzacji  $x = a(t-\sin t), ~ y = a(1-\cos t)$
        oraz osią $OX$.
\end{itemize}

\bigskip

\textbf{Zadanie 3.1} Oblicz długość podanych krzywych:

\begin{itemize}
    \item $x^{\frac{2}{3}} + y^{\frac{2}{3}} = a^{\frac{2}{3}}$ 
    \item $y = 2 \sqrt{x}$ dla $x$ od 0 do 1
    \item  $x = \frac{c^2}{a} \cos^3 t,~ y = \frac{c^2}{b} \sin^3 t,~ (c^2
        = a^2 - b^2)$
\end{itemize}

\bigskip

\textbf{Zadanie 4} Obliczyć objętość i pole powierzchni torusa powstałego
przez obrót koła $K$ 

\[
K: x^2 + (y-R)^2 \le r^2 \quad (R>r>0)
\]
dookoła osi $OX$.

\bigskip

\textbf{Zadanie 4.1} Obliczyć objętości brył powstałych przez obórt
poniższych krzywych wokół osi $OX$:

 \begin{itemize}
    \item $y = \sin x$ dla $0\le x \le \pi$
    \item $y = ax - x^2$ dla $x$ takich, że $y(x) >0$
    \item $\frac{x^2}{a^2} + \frac{y^2}{b^2} = 1$ (elipsoida obrotowa)
\end{itemize}

\bigskip

\textbf{Zadanie 5}$^\ast$ Udowodnić, że jeżeli funkcja $f : [0,\infty) →
\mathbb{R}$ jest ciągła, malejąca i dodatnia to

\[
\lim_{t\to 0^+} t \sum_{n=1}^\infty f(tn) = \int_{0}^\infty f(x) dx
.\] 

Następnie korzystając z tej równości wyznacz granice:

\[
\lim_{t\to 0^+} \sum_{n=1}^\infty \frac{t}{1+(nt)^2} 
\quad , \quad
\lim_{t\to 0^+} t \sum_{n=1}^\infty ne^{-tn^2}
.\] 

\bigskip

\textbf{Zadanie 6} Wykaż, że funkcja $\chi: [-1,1] \to \mathbb{R}$ zadana
wzorem

 \[
\chi(x) = 
\begin{cases}
    0 & \text{dla~} -1 \le x \le 0 \\
    1 & \text{dla~} 0 < x \le  1
\end{cases}
\] 
jest całkowalna w sensie Riemanna (choć nie jest ciągła...).

\bigskip

\end{document}

