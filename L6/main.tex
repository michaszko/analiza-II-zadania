%! TeX program = lualatex
%! TEX options = -synctex=1 -interaction=nonstopmode -file-line-error --shell-escape "%DOC%"
\documentclass[a4paper,11pt]{article}
\usepackage{polski}
% \usepackage{libertine}
\usepackage[top=1.5cm, bottom=1.5cm, left=2.5cm, right=2.5cm]{geometry}
\usepackage{url}
\usepackage{graphicx}
\usepackage{amsfonts}
\usepackage{amsmath}
\usepackage{enumitem}
\usepackage{multicol}
\usepackage{fontspec}

\newfontfamily\DejaSans{DejaVu Sans}

\everymath{\displaystyle}

\setlength\parindent{0.5pt} 

\begin{document}

\begin{center}
  {\large\textbf{Lista 6}}
\end{center}

\hrulefill
\begin{center}
    \textbf{Zadania do deklaracji (poniedziałek)}
\end{center}

\bigskip

\textbf{Zadanie 1} Dla $f:[a,b] \to \mathbb{R}$ określamy:

\begin{itemize}
    \item \textit{długość wykresu} $f$, o ile $f$ jest klasy $C^1$, wzorem 
    $$\int_a^b \sqrt{1+(f'(x))^2} ~dx;$$
    \item \textit{pole powierzchni obrotowej} powstałej przez obrót wykresu $f$ 
    (zawartego w płaszczyźnie $XY$) wokół osi $X$ w przestrzeni $XYZ$, 
    o ile $f$ jest klasy $C^1$, wzorem
    $$2\pi \int_a^b f(x) \sqrt{1+(f'(x))^2}~dx;$$
    \item \textit{objętość bryły obrotowej ograniczonej powyższą powierzchnią 
    obrotową i płaszczyznami "$x=a$" oraz "$x=b$", o ile $f$ jest ciągła, wzorem}
    $$\pi \int_a^b (f(x))^2 ~dx.$$
\end{itemize}

W oparciu o powyższe wzory oblicz:
\begin{enumerate}
    \item  długość okręgu o promieniu $r$,
    \item objętość kuli o promieniu $r$,
    \item pole powierzchni sfery o promieniu $r$,
    \item  objętość walca obrotowego o promieniu podstawy $r$ i wysokości $h$,
    \item pole powierzchni bocznej walca obrotowego o promieniu podstawy $r$ i wysokości $h$,
    \item objętość stożka obrotowego o promieniu podstawy $r$ i wysokości $h$,
    \item pole powierzchni bocznej stożka obrotowego o promieniu podstawy $r$ i wysokości $h$.
\end{enumerate}

\textit{Proszę zrobić 4 wybrane przez was zadania z wymienionych powyżej.} 

\bigskip

\end{document}

