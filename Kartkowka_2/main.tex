%! TeX program = lualatex
%! TEX options = -synctex=1 -interaction=nonstopmode -file-line-error --shell-escape "%DOC%"
\documentclass[a4paper,11pt]{article}
\usepackage{polski}
% \usepackage{libertine}
\usepackage[top=1.5cm, bottom=1.5cm, left=2.5cm, right=2.5cm]{geometry}
\usepackage{url}
\usepackage{graphicx}
\usepackage{amsfonts}
\usepackage{amsmath}
\usepackage{enumitem}
\usepackage{multicol}
\usepackage{fontspec}
\usepackage{bm}

\newfontfamily\DejaSans{DejaVu Sans}

\everymath{\displaystyle}
\newcommand{\dm}[1]{\displaystyle{#1}}
\newcommand{\RR}{\mathbb{R}}

\setlength\parindent{0.5pt} 

\begin{document}

\begin{center}
  {\large\textbf{Przykładowa kwartkówa}}
\end{center}

\hrulefill

\bigskip

\textbf{Zadanie 1} Policz poniższe całki nieoznaczone:

\[
\int \frac{2x+3}{2x+1} dx 
\quad \quad 
\int \frac{e^{\frac{1}{x}}}{x^2} dx
\quad \quad
\int e^{\sqrt{x}}  dx
.\] 

\bigskip

\textbf{Zadanie 2} Znajdź pole powierzchni figury ograniczonej przez
krzywą $y(x) = a \cosh \frac{x}{a}$ oraz prostą $y(x) = \frac{a}{2e} (e^2
+ 1)$.

\bigskip

\textbf{Zadanie 3} Znajdź długość krzywej $y(x) = \ln x$ pomiędzy $x =
\sqrt{3} $ oraz $x=\sqrt{8}$ 

\bigskip

\textbf{Zadanie 4} Znajdź poniższe granice lub pokaż, że nie istnieją:

\[
\lim_{(x,y)\to (0,0)} \frac{2xy}{x^2 + y^2}
\quad \quad
\lim_{(x,y) \to (0,0)} \frac{2xy^2}{x^2 + y^2}
.\] 

\hrulefill

\begin{center}
    \textbf{Zadania dodatkowe}
\end{center}

Na prawdziwej kartkówce dodam tutaj kilka zadań dodatkowych, dla tych
którzy skończą wcześniej.

\end{document}

