%! TeX program = lualatex
%! TEX options = -synctex=1 -interaction=nonstopmode -file-line-error --shell-escape "%DOC%"
\documentclass[a4paper,11pt]{article}
\usepackage{polski}
% \usepackage{libertine}
\usepackage[top=1.5cm, bottom=1.5cm, left=2.5cm, right=2.5cm]{geometry}
\usepackage{url}
\usepackage{graphicx}
\usepackage{amsfonts}
\usepackage{amsmath}
\usepackage{enumitem}
\usepackage{multicol}

\everymath{\displaystyle}

\setlength\parindent{0.5pt} 

\begin{document}

\begin{center}
  {\large\textbf{Lista 2b}}
\end{center}

\textbf{Zadanie 0} Udowodnij, że szereg $\sum_{n=0}^\infty a_n$ 
(gdzie ciąg wyrazów jest nierosnący oraz $a_n \geq 0$) jest zbieżny 
wtedy i tylko wtedy kiedy szereg $\sum_{n=0}^\infty 2^n a_{2^n}$ jest zbieżny.

\bigskip

\textbf{Zadanie 1} Znajdź promień zbieżności podanych szeregów:

\begin{multicols}{3}
    \begin{enumerate}
        \item $\sum_{n=1}^\infty \frac{1}{n^x}$
        \item $\sum_{n=1}^\infty \frac{(-1)^{n+1}}{n^x}$
        \item $\sum_{n=1}^\infty \frac{\sin(2n-1) x}{(2n-1)^2}$
        \item $\sum_{n=1}^\infty 2^n \sin \frac{x}{3^n}$
        \item $\sum_{n=1}^\infty \frac{n!}{x^n}$
        \item $\sum_{n=1}^\infty \frac{1}{n! x^n}$
        \item $\sum_{n=1}^\infty \frac{1}{(2n-1)x^n}$
        \item $\sum_{n=1}^\infty \frac{\sqrt{n}}{(x-2)^n}$
        \item $\sum_{n=1}^\infty \frac{n^n}{x^{n^n}}$
    \end{enumerate}
\end{multicols}

\bigskip

\textbf{Zadanie 2} Znajdź promień zbieżności podanych szeregów potęgowych i 
sprawdź zbieżność na granicach promienia zbieżności:

\begin{multicols}{3}
    \begin{enumerate}
        \item $\sum_{n=0}^\infty x^n$
        \item $\sum_{n=1}^\infty \frac{x^n}{n 2^n}$
        \item $\sum_{n=1}^\infty \frac{x^{2n-1}}{2n-1}$
        \item $\sum_{n=1}^\infty \frac{2^{n-1} x^{2n-1}}{(4n-3)^2}$
        \item $\sum_{n=0}^\infty \frac{x^n}{n!}$
        \item $\sum_{n=0}^\infty n! x^n$
        \item $\sum_{n=1}^\infty \frac{x^n}{n^n}$
        \item $\sum_{n=1}^\infty \frac{n! x^n}{n^n}$
        \item $\sum_{n=1}^\infty \frac{x^{n-1}}{n 3^n \ln n}$
        \item $\sum_{n=0}^\infty x^{n!}$
        \item $\sum_{n=1}^\infty \frac{(x-1)^{2n}}{n 9^n}$
        \item $\sum_{n=1}^\infty \frac{(x-2)^n}{(2n-1) 2^n}$
    \end{enumerate}
\end{multicols}

\bigskip

\textbf{Zadanie 3} Znajdź sumy poniższych szeregów:

\begin{multicols}{3}
    \begin{enumerate}
        \item $\sum_{n=0}^\infty (n+1) x^n$
        \item $\sum_{n=1}^\infty \frac{x^n}{n}$
        \item $\sum_{n=1}^\infty (-1)^{n-1}\frac{x^n}{n}$
        \item $\sum_{n=1}^\infty \frac{n}{x^n}$
        \item $\sum_{n=1}^\infty \frac{x^{4n-3}}{4n-3}$
        \item $\sum_{n=1}^\infty \frac{2n-1}{2^n}$
    \end{enumerate}
\end{multicols}

\hrulefill

\textbf{Zadanie 4} Dwa pociągi znajdują się w odległości $d$ od siebie i poruszają się
z prędkością $v$ w swoją stronę (idą na czołowe zderzenie). Pomiędzy tymi pociągami 
lata mucha z prędkością $v_M$, w taki sposób że jeśli doleci do pociągu to od razu 
zawraca i leci do drugiego pociągu. Jaką drogę pokona mucha do momentu zderzenia 
pociągów? Jak się zmieni wynik, jeśli pociągi będą poruszały się z innymi prędkościami?

\bigskip

\textbf{Zadanie 5} Wyobraźmy sobie "pół-nieskończony" 
\footnote{Taki, który ciągnie się w nieskończoność tylko w jedną stronę -- załóżmy że w prawo} 
ciąg ładunków elektrycznych $q$
ustawionych w jednej lini w odległości $d$ od siebie. Jaka siła wypadkowa będzie 
działała na najbardziej lewy skrajny ładunek. Siła pomiędzy ładunkami 
$F = k \frac{q^2}{d^2}$, gdzie $k$ to pewna stała.

\bigskip

\textbf{Zadanie $6^\ast$} Wyobraźmy sobie, że żyjemy w świecie dwuwymiarowym. Nagle 
stajemy przed "pół-nieskończonym" ciągiem latarni, które emitują światło z tą samą 
jasnością (mocą) $I$, i które są rozstawione w jednej lini w odległości $d$. 
Jaka będzie 
wypadkowa jasność (moc), którą będę widział? Jak się zmieni wynik jeśli przejdę do 
świata trójwymiarowego?

\end{document}

