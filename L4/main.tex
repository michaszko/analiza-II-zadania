%! TeX program = lualatex
%! TEX options = -synctex=1 -interaction=nonstopmode -file-line-error --shell-escape "%DOC%"
\documentclass[a4paper,11pt]{article}
\usepackage{polski}
% \usepackage{libertine}
\usepackage[top=1.5cm, bottom=1.5cm, left=2.5cm, right=2.5cm]{geometry}
\usepackage{url}
\usepackage{graphicx}
\usepackage{amsfonts}
\usepackage{amsmath}
\usepackage{enumitem}
\usepackage{multicol}

\everymath{\displaystyle}

\setlength\parindent{0.5pt} 

\begin{document}

\begin{center}
  {\large\textbf{Lista 4}}
\end{center}

\hrulefill
\begin{center}
    \textbf{Zadania do deklaracji (poniedziałek)}
\end{center}

\bigskip

\textbf{Zadanie 1} Oblicz następujące całki nieoznaczone:

\begin{multicols}{3}
    \begin{enumerate}
        \item $\int \sin x e^x dx$
        \item $\int \frac{x^4}{1+x^2} dx$
        \item $\int x^3 \ln x dx$
    \end{enumerate}
\end{multicols}

\bigskip

\textbf{Zadanie 2} Oblicz następujące całki oznaczone:

\begin{multicols}{3}
    \begin{enumerate}
        \item $\int \limits_0^1 x^3 \sqrt{7+x^4} dx$
        \item $\int \limits_0^1 \frac{e^t}{e^{2t} + e^t + 1} dt$
        \item $\int \limits_0^1 \frac{s}{1+s^4} ds$
    \end{enumerate}
\end{multicols}

\bigskip

\textbf{Zadanie 3} Oblicz $\smallint_0^\infty x^n e^{-x}$ dla
każdego $n \in \mathbb{N}_0$.

\textit{Wskazówka: sprawdź w internecie "Funkcja gamma Eulera"}

\bigskip

\textbf{Zadanie 4} Niech $f: [0,1] \to (0,\infty)$ będzie funkcją 
ciągłą. Udowodnić że 

$$\int \limits_0^1 \frac{f(x)}{f(x) + f(1-x)} dx = \frac{1}{2}$$

a następnie obliczyć

$$\int \limits_0^{\frac{\pi}{2}} \frac{\sqrt{\sin x}}{\sqrt{\sin x} + \sqrt{\cos x}} dx.$$

\bigskip

\hrulefill

\begin{center}
    \textbf{Zadania na zajęcia}
\end{center}

\bigskip

\textbf{Zadanie 5} Stosując wzór na całkowanie przez części 
obliczyć całki nieoznaczone:

\begin{multicols}{4}
    \begin{enumerate}
        \item $\int x \cdot \arctan x dx$,
        \item $\int (\sin x)^4 dx$,
        \item $\int e^{3x} \sin (4x) dx$
        \item $\int x e^x \sin x dx$
    \end{enumerate}
\end{multicols}

\bigskip

\textbf{Zadanie 6} Obliczyć całki nieoznaczone z funkcji wymiernych:

\begin{multicols}{3}
    \begin{enumerate}
        \item $\int \frac{x^4 + x^3 + x^2 + 4x -1}{x^3 - x} dx$,
        \item $\int \frac{2x+5}{(x^2 + 4x + 5)^2} dx$,
        \item $\int \frac{8}{x^4+4} dx$
    \end{enumerate}
\end{multicols}

\bigskip

\textbf{Zadanie 7} . Stosując wzór na całkowanie przez podstawienie obliczyć całki nieoznaczone:

\begin{multicols}{4}
    \begin{enumerate}
        \item $\int \frac{x^2}{\sqrt{1+x^3}} dx$,
        \item $\int \frac{dx}{e^{3x}+1}$,
        \item $\int \sqrt{\frac{x+1}{x-1}} dx$
        \item $\int (\sin x)^5$
    \end{enumerate}
\end{multicols}

\bigskip

\textbf{Zadanie 8} Obliczyć całki nieoznaczone wykorzystując podstawienie trygonometryczne (lub inaczej):

\begin{multicols}{3}
    \begin{enumerate}
        \item $\int \frac{dx}{\sin x}$,
        \item $\int \frac{dx}{5+3\cos x}$,
        \item $\int \frac{1+\sin x \cos x}{(2+\cos^2 x)(1+\sin^2 x)} dx$
    \end{enumerate}
\end{multicols}

\textbf{Zadanie 9} Obliczyć całki nieoznaczone z funkcji zawierających pierwiastek z trójmianu kwadratowego:

\begin{multicols}{3}
    \begin{enumerate}
        \item $\int \sqrt{3-2x-x^2} dx$
        \item $\int \frac{x^2}{\sqrt{x^2-4x + 3}} dx$
        \item $\int \frac{\sqrt{x^2 + 4x + 9}}{x^2} dx$
    \end{enumerate}
\end{multicols}

\textbf{Zadanie 10} Pokaż w jednej linijce 

$$\int \limits_{-L}^L \sin nx \cos mx dx = 0$$

dla $L>0$ i $n,m \in \mathbb{Z}$.

\end{document}

