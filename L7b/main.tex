%! TeX program = lualatex
%! TEX options = -synctex=1 -interaction=nonstopmode -file-line-error --shell-escape "%DOC%"
\documentclass[a4paper,11pt]{article}
\usepackage{polski}
% \usepackage{libertine}
\usepackage[top=1.5cm, bottom=1.5cm, left=2.5cm, right=2.5cm]{geometry}
\usepackage{url}
\usepackage{graphicx}
\usepackage{amsfonts}
\usepackage{amsmath}
\usepackage{enumitem}
\usepackage{multicol}
\usepackage{fontspec}

\newfontfamily\DejaSans{DejaVu Sans}

\everymath{\displaystyle}
\newcommand{\dm}[1]{\displaystyle{#1}}

\setlength\parindent{0.5pt} 

\begin{document}

\begin{center}
  {\large\textbf{Lista 7b}}
\end{center}

\textbf{Zadanie 1} Zbadaj ciągłość funkcji $f: \mathbb{R}^d \to
\mathbb{R}$ zadanej wzorem

\[
f(x) = 
\begin{cases}
    w(x) & \text{~dla~} x\neq 0\\
    0 & \text{~dla~} x =0
\end{cases}
,\]

gdy $w(x)$ dla  $x\neq 0$ jest zadane wzorem:

\begin{itemize}
    \item $\frac{x_1^2 x_2^2}{x_1^2 x_2^2 + (x_1-x_2)^2}$, $d=2$;
    \item  $\frac{x_1 x_3 + x_2 x_3}{x_1^2 + x_2^2 + x_3^2}$, $d=3$;
    \item  $\frac{x_1^2 x_2}{x_1^2 x_2^2 + x_1^2 + x_2^2}$, $d=2$;
    \item  $\frac{x_1^2 x_2}{x_1^4 + x_2^2}$, $d=2$,
    \item $\frac{3x_1^2 x_2^2 + x_1 x_2^4}{(x_1^2 + x_2^2)^2}$, $d=2$.
\end{itemize}

\bigskip

\textbf{Zadanie 2} Pokaż, że w dowolnej przestrzeni metrycznej, każda
kula otwarta o $r>0$ jest zbiorem otwartym, a każdy jednopunktowy zbiór
jest zbiorem domkniętym.

\bigskip

\textbf{Zadanie 3} Udowodnij, że jeśli $d(\cdot, \cdot)$ jest metryką w
przestrzeni  $S$ to metryka  $d'(\cdot, \cdot)$ w  $S$ dana przez:

\begin{itemize}
    \item $d'(x,y) = \min\{1, d(x,y)\}$
    \item  $d'(x,y) = \frac{d(x,y)}{1+d(x,y)}$
\end{itemize}

też jest "prawdziwą" metryką. W pierwszym przypadku pokaż, że kula
$K(p,\epsilon)$ o promieniu  $\epsilon < 1$ i środku $p$ jest taka sama w
przypadku $d$ oraz $d'$. W drugim przypadku pokaż, że $K(p,\epsilon)$ w
$(S,d)$ jest także kulą $K(p,\epsilon')$ w $(S,d')$ o promieniu
$\epsilon' = \frac{\epsilon}{1+\epsilon}$.

\end{document}

