%! TeX program = lualatex
%! TEX options = -synctex=1 -interaction=nonstopmode -file-line-error --shell-escape "%DOC%"
\documentclass[a4paper,11pt]{article}
\usepackage{polski}
% \usepackage{libertine}
\usepackage[top=1.5cm, bottom=1.5cm, left=2.5cm, right=2.5cm]{geometry}
\usepackage{url}
\usepackage{graphicx}
\usepackage{amsfonts}
\usepackage{amsmath}
\usepackage{enumitem}
\usepackage{multicol}
\usepackage{fontspec}
\usepackage{bm}

\newfontfamily\DejaSans{DejaVu Sans}

\everymath{\displaystyle}
\newcommand{\dm}[1]{\displaystyle{#1}}
\newcommand{\RR}{\mathbb{R}}
\newcommand{\CC}{\mathbb{C}}

\setlength\parindent{0.5pt} 

\begin{document}

\begin{center}
  {\large\textbf{Lista 10}}
\end{center}

\hrulefill
\begin{center}
    \textbf{Zadania do deklaracji (piątek)}
\end{center}

\bigskip

Dodam za niedługo

\bigskip

\hrulefill

\bigskip

\textbf{Zadanie 5} Czy istnieje niemierzalny zbiór $A \subset \RR$ taki że 
$B= \{x \in A : x \text{~jest liczbą niewymierną}\}$ jest mierzalny?

\bigskip

\textbf{Zadanie 6} Udowodnić, że jeśli funkcja $f : [0,\infty) \to \RR$
jest różniczkowalna, to jej pochodna $f'$ jest mierzalna.

\bigskip

\textbf{Zadanie 7} Obliczyć następujące całki:

\begin{itemize}
  \item $\int_K x_1 x_2 d \lambda_3(x), \quad \text{gdzie} \quad K = \{x
    \in \RR^3: x_1,x_2,x_3 > 0,  \|x\| < 1\} $
  \item $\int_{R} x_1 d \lambda_2(x), \quad$ gdzie $R$ jest
    równoległobokiem w $\RR^2$ o wierzchołkach $(0,0), (2,1), (1,1),
    (3,2)$
\end{itemize}

\bigskip

\textbf{Zadanie 8} Wyznaczyć następujące granice 

\begin{itemize}
  \item $\lim_{n \to \infty} \int_{W} \sqrt[n]{x_1x_2}~ d\lambda_2(x),
    \quad \text{gdzie} \quad W = \{x \in \RR^3: \|x\|\le 1, 0\le
    x_1<x_2\} $
  \item $\lim_{n \to \infty} \int_{A} \left( 1+\frac{x+y}{n} \right)^n ~
    d\lambda_2(x), \quad \text{gdzie} \quad A = \{(x,y) : x>0, x+y>0\} $
\end{itemize}


\end{document}

