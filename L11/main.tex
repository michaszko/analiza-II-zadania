%! TeX program = lualatex
%! TEX options = -synctex=1 -interaction=nonstopmode -file-line-error --shell-escape "%DOC%"
\documentclass[a4paper,11pt]{article}
\usepackage{polski}
% \usepackage{libertine}
\usepackage[top=1.5cm, bottom=1.5cm, left=2.5cm, right=2.5cm]{geometry}
\usepackage{url}
\usepackage{graphicx}
\usepackage{amsfonts}
\usepackage{amsmath}
\usepackage{enumitem}
\usepackage{multicol}
\usepackage{fontspec}
\usepackage{bm}

\newfontfamily\DejaSans{DejaVu Sans}

\everymath{\displaystyle}
\newcommand{\dm}[1]{\displaystyle{#1}}
\newcommand{\RR}{\mathbb{R}}
\newcommand{\CC}{\mathbb{C}}

\setlength\parindent{0.5pt} 

\begin{document}

\begin{center}
  {\large\textbf{Lista 11}}
\end{center}

\hrulefill
\begin{center}
    \textbf{Zadania do deklaracji (piątek)}
\end{center}

\bigskip

\textbf{Zadanie 1} Obliczyć $\lim_{n\to +\infty} \int_{0}^\infty
\frac{1}{1+x^n}d\lambda_1(x)$

\bigskip

\textbf{Zadanie 2} Obliczyć granicę 
$\lim_{n\to +\infty} \int_0^\infty
\frac{(\frac{1}{2}+x)^n}{(\frac{1}{2}+x)^n + 1} e^{-2x} d\lambda_1(x)$

\bigskip

\textbf{Zadanie 3} Oblicz $\int_W d \lambda_3(x)$ gdzie  $W = \{x \in
\RR^3: x_1,x_2,x_3 \ge 0, x_1+x_2+x_3 \le 1\} $ (objętość czworościanu).

\bigskip

\textbf{Zadanie 4} Oblicz całkę  $\int_{R} d \lambda_2(x), \quad$
gdzie $R$ jest równoległobokiem w $\RR^2$ o wierzchołkach $(0,0), (2,1),
(1,1), (3,2)$ (pole równoległoboku).

\bigskip

\hrulefill

\bigskip

\textbf{Zadanie 5} Czy istnieje niemierzalny zbiór $A \subset \RR$ taki że 
$B= \{x \in A : x \text{~jest liczbą niewymierną}\}$ jest mierzalny?

\bigskip

\textbf{Zadanie 6} Udowodnić, że jeśli funkcja $f : [0,\infty) \to \RR$
jest różniczkowalna, to jej pochodna $f'$ jest mierzalna.

\bigskip

\textbf{Zadanie 7} Obliczyć następujące całki:

\begin{itemize}
  \item $\int_K x_1 x_2 d \lambda_3(x), \quad \text{gdzie} \quad K = \{x
    \in \RR^3: x_1,x_2,x_3 > 0,  \|x\| < 1\} $
  \item $\int_{R} x_1 d \lambda_2(x), \quad$ gdzie $R$ jest
    równoległobokiem w $\RR^2$ o wierzchołkach $(0,0), (2,1), (1,1),
    (3,2)$
  \item $\int_{[0,1]\times [0,2]} e^{x_1 + x_2} dx$ 
  \item $\int_{[-1,1]\times [0,1]} (x_1 + x_2)^{2222} dx$ 
  \item $\int_{T} d\lambda_2(x,y),\quad$ gdzie $T$ to trójkąt pełny na
    płaszczyźnie $\RR^2$ o wierzchołkach $(0,0),(a,0),(c,h)$, gdzie
    $a,h,c>0$.
  \item  $\int_A d\lambda_2(x,y), \quad$ gdzie  $A=\{(x_1,x_2) \in \RR^2:
    x_1 \ge 0, x_1^2 \le x_2 \le \sqrt{x_1} \} $
\end{itemize}

\bigskip

\textbf{Zadanie 8} Wyznaczyć następujące granice 

\begin{itemize}
  \item $\lim_{n \to \infty} \int_{W} \sqrt[n]{x_1x_2}~ d\lambda_2(x),
    \quad \text{gdzie} \quad W = \{x \in \RR^3: \|x\|\le 1, 0\le
    x_1<x_2\} $
  \item $\lim_{n \to \infty} \int_{A} \left( 1+\frac{x+y}{n} \right)^n ~
    d\lambda_2(x), \quad \text{gdzie} \quad A = \{(x,y) : x>0, x+y>0\} $
  \item $\lim_{n \to \infty} \int_{0}^\infty \frac{x^n +2}{x^n + 1}
    e^{-x} dx$
  \item $\lim_{n\to \infty} \int_{\RR^2} \frac{x^n + y^n}{1+x^n + y^n}
    e^{-x-y} d\lambda_2(x,y)$
  \item $\lim_{n\to \infty} \int_1^\infty \frac{\ln(1+nx)}{1+x^2\ln n} dx$
\end{itemize}


\end{document}

