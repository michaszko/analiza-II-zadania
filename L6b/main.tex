%! TeX program = lualatex
%! TEX options = -synctex=1 -interaction=nonstopmode -file-line-error --shell-escape "%DOC%"
\documentclass[a4paper,11pt]{article}
\usepackage{polski}
% \usepackage{libertine}
\usepackage[top=1.5cm, bottom=1.5cm, left=2.5cm, right=2.5cm]{geometry}
\usepackage{url}
\usepackage{graphicx}
\usepackage{amsfonts}
\usepackage{amsmath}
\usepackage{enumitem}
\usepackage{multicol}
\usepackage{fontspec}

\newfontfamily\DejaSans{DejaVu Sans}

\everymath{\displaystyle}

\setlength\parindent{0.5pt} 

\begin{document}

\begin{center}
  {\large\textbf{Lista 6.1}}
\end{center}

\textbf{Zadanie 1}  Niech $P$ będzie puntem na paraboli $y=x^2$ innym niż
punkt  $(0,0)$. Normalna do paraboli w punkcie  $P$ przecina tę parabolę
w innym punkcie $Q$. Znajdź punkt $P$, tak że długość paraboli pomiędzy
punktem $P$ i $Q$ jest minimalna.

\bigskip

\textbf{Zadanie 1.1} To samo co wyżej tylko znajdź punkt $P$, który
minimalizuje obwód figury ograniczonej parabolą oraz normalną.

\bigskip

\textbf{Zadanie 2} Obracamy figurę ograniczoną krzywymi $y=x^2$ oraz
$y=x$ wzdłuż prostej o równaniu  $y=x$. Jaka jest objętość takiej figury?
Jaka jest jest powierzchnia boczna?

\bigskip 
\textbf{Zadanie 3} Tadek, jako prawdziwy matematyk, chce wysłać swojej
ukochanej walentynkę w kształcie serca o równaniu parametrycznym $x(t) =
\cos t (1-\sin t),~ y(t) = \sin t (1-\sin t)$. Chce je włożyć do
prostokątnego pudełka. Jakie muszą być minimalne wymiary tego pudełka,
żeby walentynka zmieściła się cała do pudełka? (Jest tu kilka opcji
pakowania, wybierz najprostszą dla siebie.)

\bigskip

\textbf{Zadanie 3.1$^\ast$} To samo co wyżej, tylko walentynka ma równanie
$(x^2 + y^2 - 1)^3 = x^2 y^3$.

\bigskip

\textbf{Zadanie 4$^\ast$} Znajdź pole figury ograniczoną krzywą Lissajous o
parametryzacji $x(t)  = \sin(2t), ~y(t) = \sin(3t)$.

\bigskip

\hrulefill

\bigskip

\textbf{Zadanie 5} Znajdź pole powierzchni pomiędzy 
\begin{itemize}
    \item krzywą $y=x(x-1)(x-2)$ a osią  $OX$;
    \item hiperbolą $xy = m^2$, liniami  $x=a$, $x=3a$ $a>0$ oraz osią
        $OX$;
    \item krzywą  $y^2 = 2px$ oraz  $x^2 = 2py$
\end{itemize}

\bigskip

\textbf{Zadanie 6} Znajdź długość zwisającego luźno kabla pomiędzy
punktami $A(0,a)$ a  $B(b,h)$ o równaniu  $y=a\cosh \frac{x}{a}$.

\bigskip

\textbf{Zadanie 7} Policz długość krzywej o równaniu $y^2 = x^3$ od
początku osi współrzędnych do  $x=4$.

\bigskip

 \textbf{Zadanie 8} Policz objętość bryły obrotowej powstałej w wyniku
 obrotu krzywej $y = \sin^2 x$ wokół osi $OX$ na przedziale  $x=0$ do
 $x=\pi$. Następnie opakowujemy tą bryłę ciasno sferą. Jaki jest stosunek
 objętości tej kuli do objętości bryły "sinusowej"?

\end{document}

