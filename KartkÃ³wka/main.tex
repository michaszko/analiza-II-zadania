%! TeX program = lualatex
%! TEX options = -synctex=1 -interaction=nonstopmode -file-line-error --shell-escape "%DOC%"
\documentclass[a4paper,11pt]{article}
\usepackage{polski}
% \usepackage{libertine}
\usepackage[top=1.5cm, bottom=1.5cm, left=2.5cm, right=2.5cm]{geometry}
\usepackage{url}
\usepackage{graphicx}
\usepackage{amsfonts}
\usepackage{amsmath}
\usepackage{enumitem}
\usepackage{multicol}

\everymath{\displaystyle}

\setlength\parindent{0.5pt} 

\begin{document}

\thispagestyle{empty}

\begin{center}
  {\large\textbf{Kartkówka}}\\
  {\small 27 marca 2023}
\end{center}

\hrulefill

\bigskip

\textbf{Zadanie 1} Oblicz następującą granicę:

\[
\lim_{x \to 0} \frac{e^{3x} + \ln(1-2x) - \sin x - \cos x}{-1 +\cos 5x}
.\]

\bigskip

\textbf{Zadanie 2} Ile wyrazów szeregu 

\[
e^x = 1 + \frac{x}{1!} + \frac{x^2}{2!} + ... 
\]

trzeba wziąć, żeby policzyć liczbe $e$ z dokładnością do 
3 miejsca po przecinku.\\

\textit{Dla chcących sobie utrudnić to zadanie:} załóż, że nie wiesz o 
funkcji $e^x$ nic oprócz 1) $e^0 = 1$; 2) $(e^x)' = e^x$; 
3) $ \forall_x~~e^x > 0$.

\bigskip

\textbf{Zadanie 3} Oblicz przedział i promień zbieżności 
szeregów potęgowych

\[
\sum_{n=0}^\infty \frac{n^{2n+1}}{4^{3n}}(2x+17)^n
\quad\quad\quad
\sum_{n=0}^\infty (2 + (-1)^{n})^{3n+2} x^{2n+1}
\]

\bigskip

\textbf{Zadanie 4} Znajdź sumę następującego szeregu:

$$\sum_{n=1}^\infty \frac{n}{2^{2n-1}} x^{n-1}$$

\bigskip

\textbf{Zadanie 5} Znaleźć postać ogólną funkcji $f$ tożsamościowo
spełniającej równanie

$$(x^2 + 1) f'' - 4x f' + 6f = 0$$

zakładając, że $f$ jest sumą pewnego szeregu potęgowego o środku w 
$x_0 = 0$.

\bigskip

\textbf{Zadanie 6} Oblicz całkę nieoznaczoną

$$\int \frac{1}{e^x + 1} dx$$

% \hrulefill

\end{document}

