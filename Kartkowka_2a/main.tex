%! TeX program = lualatex
%! TEX options = -synctex=1 -interaction=nonstopmode -file-line-error --shell-escape "%DOC%"
\documentclass[a4paper,11pt]{article}
\usepackage{polski}
% \usepackage{libertine}
\usepackage[top=1.5cm, bottom=1.5cm, left=2.5cm, right=2.5cm]{geometry}
\usepackage{url}
\usepackage{graphicx}
\usepackage{amsfonts}
\usepackage{amsmath}
\usepackage{enumitem}
\usepackage{multicol}
\usepackage{fontspec}
\usepackage{bm}

\newfontfamily\DejaSans{DejaVu Sans}

\everymath{\displaystyle}
\newcommand{\dm}[1]{\displaystyle{#1}}
\newcommand{\RR}{\mathbb{R}}

\setlength\parindent{0.5pt} 

\begin{document}

\begin{center}
  {\large\textbf{Kartkówka 2}}
\end{center}

\hrulefill

\bigskip

\textbf{Zadanie 1} Policz poniższe całki nieoznaczone:

\[
\int \frac{x^2+1}{x-1} dx 
\quad \quad 
\int \frac{1}{1-e^{-x}} dx
\quad \quad
\int \frac{x}{e^x}  dx
.\] 

\bigskip

\textbf{Zadanie 2} Policz pole powierzchni figury ograniczonej przez
krzywe $y^2=2px$ i  $x^2 = 2py$. Wskazówka: narysować!

\bigskip

\textbf{Zadanie 3} Znajdź długość krzywej danej wzorem $y(x) =
\frac{1}{4} x^2 - \frac{1}{2} \ln x$ pomiędzy $x = 1$ oraz $x=e$.

\bigskip

\textbf{Zadanie 4} Znajdź poniższe granice lub pokaż, że nie istnieją:

\[
\lim_{(x,y)\to (0,0)} \frac{x + 2xy + y}{x^2 + y^2}
\quad \quad
\lim_{(x,y) \to (0,0)} \frac{5x^2y^2}{x^2 + y^2}
.\] 

\hrulefill

\begin{center}
    \textbf{Zadania dodatkowe}
\end{center}

\textbf{Zadanie 1} Policz pola powierzchni \textbf{obu} części na które
jest podzielone koło o równaniu {$x^2 + y^2 \le 8$} przez parabolę  $y^2 =
2x$. Wskazówka: narysować.

\bigskip

\textbf{Zadanie 2} Rozważmy funkcję $f(x) = x^n$ na przedziale  $[0,1]$.
Niech  $x_0 \in (0,1)$. Określmy  przez  $A(x_0)$ pole powierzchni
ogranioczonej przez proste $y=0$ i  $x=x_0$ oraz krzywą $f(x)$, a przez
$B(x_0)$  pole powierzchni ograniczonej przez $x=x_0$ i $y=1$ oraz krzywą
 $f(x)$. Dla jakiego  $x_0$ zachodzi  $A(x_0) = B(x_0)$?

 \bigskip

 \textbf{Zadanie 3} Policz objętość figury powstałej na skutek obrotu
 wzdłóż prostej $x=a$ części paraboli  $y^2=4ax$, odciętej przez prostą
  $x=a$.

\end{document}

