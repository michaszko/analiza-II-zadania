%! TeX program = lualatex
%! TEX options = -synctex=1 -interaction=nonstopmode -file-line-error --shell-escape "%DOC%"
\documentclass[a4paper,11pt]{article}
\usepackage{polski}
% \usepackage{libertine}
\usepackage[top=1.5cm, bottom=1.5cm, left=2.5cm, right=2.5cm]{geometry}
\usepackage{url}
\usepackage{graphicx}
\usepackage{amsfonts}
\usepackage{amsmath}
\usepackage{enumitem}
\usepackage{multicol}
\usepackage{fontspec}
\usepackage{bm}

\newfontfamily\DejaSans{DejaVu Sans}

\everymath{\displaystyle}
\newcommand{\dm}[1]{\displaystyle{#1}}
\newcommand{\RR}{\mathbb{R}}

\setlength\parindent{0.5pt} 

\begin{document}

\begin{center}
  {\large\textbf{Lista 8}}
\end{center}

\hrulefill
\begin{center}
    \textbf{Zadania do deklaracji (poniedziałek)}
\end{center}

\bigskip

\textbf{Zadanie 1} Dla jakich wartości $n,m > 1$ funkcja  $f:
\mathbb{R}^2 \to \mathbb{R}$ określona wzorem 

\[
    f(x,y) = 
    \begin{cases}
        \frac{x^n + y^n}{x^m + y^m} & \text{gdy~} (x,y)  \neq  (0,0) \\
        0 & \text{gdy~} (x,y) = (0,0)
    \end{cases}
\] 

jest różniczkowalna na całej dziedzinie $\mathbb{R}^2$?

Wskazówka: funkcja jest różniczkowalna kiedy wszystkie jej pochodne
cząstkowe istnieją i są ciągłe.

\bigskip

\textbf{Zadanie 2} Wykaż, że jeżeli istnieje pochodna kierunkowa
$\frac{\partial f}{\partial \bm{v}}$, to dla dowolnego $\alpha \in
\mathbb{R}$ istnieje także  $\frac{\partial f}{\partial (\alpha \bm{v})}$
oraz  $\frac{\partial f}{\partial (\alpha \bm{v})} = \alpha \frac{\partial
f}{\partial \bm{v}}$

\bigskip

\textbf{Zadanie 3} Znajdź macierz Jakobiego i jej wyznacznik (jakobian)
funkcji $f: \mathbb{R}^2 \to \mathbb{R}^3$, $f(x,y) = (x, x+y, x \cdot
y)$ w punkcie $a=(1,7)$.

\bigskip

\textbf{Zadanie 4} Oblicz $\frac{\partial f}{\partial \bm{v}}(\bm{a})$
dla $f(x,y,z) = y z^2$, $\bm{a} = (3,1,-1)$ oraz  $\bm{v} = (-1,2,0)$.

\bigskip

\hrulefill

\bigskip

\textbf{Zadanie 5} Wyznaczyć z definicji różniczkę funkcji $f :
\mathbb{R}^2 \to \mathbb{R}, f(x, y) = xy + x^3$ , w punkcie $(1, 1)$.

\bigskip

\textbf{Zadanie 6} Znajdź wszystkie pochodne cząstkowe następujących
funkcji:

\begin{multicols}{2}
    \begin{itemize}
        \item $f(x,y) = x^3 + y^3 - 3axy$ 
        \item $f(x,y) = \frac{y}{x}$ 
        \item $f(x,y) = \sqrt{x^2 - y^2} $ 
        \item $f(x,y) = x^y$
    \end{itemize}
\end{multicols}

\bigskip

\textbf{Zadanie 7} Wykazać, że funkcja $f: \mathbb{R}^2 \to \mathbb{R}$

\[
f(x,y) =
\begin{cases}
    \frac{xy^2}{x^2 + y^2}, & \text{gdy~} (x,y) \neq (0,0) \\
    0, & \text{gdy~} (x,y) = (0,0)
\end{cases}
\] 

jest ciągła. Wyznaczyć jej pochodne cząstkowe funkcji $f$ w każdym punkcie
$(x, y) \in \mathbb{R}^2$ i pochodne kierunkowe w punkcie $(0, 0)$. Zbadać
różniczkowalność funkcji $f$ w każdym punkcie jej dziedziny.

\bigskip

\textbf{Zadanie 8} Zbadać różniczkowalność funkcji $f: \mathbb{R}^2 \to
\mathbb{R}$ oraz $g: \mathbb{R}^2 \to \mathbb{R}$ określonych wzorami

\[
f(x,y) = 
\begin{cases}
    \frac{x^3+y^3}{x^2+y^2}, & \text{gdy~} (x,y) \neq (0,0) \\
    0, & \text{gdy~} (x,y) = (0,0)
\end{cases}
\quad
g(x,y) =
\begin{cases}
    \frac{x^4+y^4}{x^2+y^2}, & \text{gdy~} (x,y) \neq (0,0) \\
    0, & \text{gdy~} (x,y) = (0,0)
\end{cases}
.\] 

\bigskip

\textbf{Zadanie 9} Nich $f: \mathbb{R} \times [0, 2\pi] \to \mathbb{R}^2$
będzie określona wzorem 

\[
f(r,\phi) = \left( r \cos\phi, r\sin \phi \right) 
.\] 

Znajdź macierz Jakobiego tego przekształcenia oraz oblicz jakobian
(wyznacznik macierzy jakobiego)

\bigskip

\textbf{Zadanie 10} Pole trapezu o podstawach $a$ oraz  $b$ i wysokości
$h$ jest dane wzorem {$S(a,b,h) = \frac{a+b}{2} h$}. Oblicz
$\frac{\partial S}{\partial a}, \frac{\partial S}{\partial b},
\frac{\partial S}{\partial h}$ i używając rysunku pokaż ich geometryczną
interpretacje.

\bigskip

\textbf{Zadanie 11} Oblicz $1.02^{3.01}$

\bigskip

\textbf{Zadanie 12} Funkcja $f$ jest określona na zbiorze $D = \{ (x,y)
    \in \RR^2: xy > -1$ wzorem 

    $$ f(x,y) = 
    \begin{cases} 
        \dfrac{\sqrt{xy+1} - 1}{y} & \text{gdy } y \neq 0, \\ \; 
        \dfrac{x}{2} & \text{gdy } y = 0. \\ 
    \end{cases} 
    $$
Zbadać różniczkowalność funkcji $f$ w punktach $(0,0)$ i $(1,0)$.
    
\textbf{Zadanie 13} Zbadać istnienie pochodnych cząstkowych w $(0,0)$ i
różniczkowalność w $(0,0)$ funkcji 

$$
\text{(a)}
\quad
f(x,y)
=
\begin{cases}
    \dfrac{2x y}{\sqrt{x^2 + y^2}} & \text{gdy } (x,y) \neq (0,0) \\ 
    0 & \text{gdy } (x,y) = (0,0) \end{cases} 
$$

$$\text{(b)} \quad 
f(x,y) = 
\begin{cases} 
    x y \sin(\frac{1}{x y})& \text{gdy } x \neq 0 \text{ i } y\neq 0 \\ 
    0 & \text{gdy } x=0 \text{ lub } y=0 
\end{cases} 
$$

\bigskip

\textbf{Zadanie 14} Wyznaczyć równanie płaszczyzny przechodzącej przez
punkt $(1,1,3)$ i stycznej do powierzchni o równaniu $z = 2x^2 + y^2$.

\bigskip

\textbf{Zadanie 15} Niech $f: \RR^2 \to  \RR$ będzie różniczkowalna oraz
$\forall_{a\in \RR^2}~  \frac{\partial f}{\partial y}(a) = 2
\frac{\partial f}{\partial x}(a)$. Wykaż, że na każdej prostej o równaniu
$2y+x= c$ funckja  $f$ jest stała.

\bigskip

\textbf{Zadanie 16} Funkcja $f: \RR^2 \to \RR$ jest różniczkowalna i dla
dowolnych $x,y \in \RR$ 
$$ y \cdot \frac{\partial f}{\partial x}(x,y) = x
\cdot \frac{\partial f}{\partial y}(x,y). $$ 
Wykazać, że funkcja $f$ jest stała na każdym okręgu o równaniu $x^2 +y^2
= r^2$.

\end{document}

