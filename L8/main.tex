%! TeX program = lualatex
%! TEX options = -synctex=1 -interaction=nonstopmode -file-line-error --shell-escape "%DOC%"
\documentclass[a4paper,11pt]{article}
\usepackage{polski}
% \usepackage{libertine}
\usepackage[top=1.5cm, bottom=1.5cm, left=2.5cm, right=2.5cm]{geometry}
\usepackage{url}
\usepackage{graphicx}
\usepackage{amsfonts}
\usepackage{amsmath}
\usepackage{enumitem}
\usepackage{multicol}
\usepackage{fontspec}

\newfontfamily\DejaSans{DejaVu Sans}

\everymath{\displaystyle}
\newcommand{\dm}[1]{\displaystyle{#1}}

\setlength\parindent{0.5pt} 

\begin{document}

\begin{center}
  {\large\textbf{Lista 8}}
\end{center}

\hrulefill
\begin{center}
    \textbf{Zadania do deklaracji (poniedziałek)}
\end{center}

\bigskip

W weekend dodam tutaj 4 zadania do deklaracji na przyszły tydzień.
\bigskip

\hrulefill

\bigskip

\textbf{Zadanie 4} Wyznaczyć z definicji różniczkę funkcji $f :
\mathbb{R}^2 \to \mathbb{R}, f(x, y) = xy + x^3$ , w punkcie $(1, 1)$.

\bigskip

\textbf{Zadanie 5} Znajdź wszystkie pochodne cząstkowe następujących
funkcji:

\begin{multicols}{2}
    \begin{itemize}
        \item $f(x,y) = x^3 + y^3 - 3axy$ 
        \item $f(x,y) = \frac{y}{x}$ 
        \item $f(x,y) = \sqrt{x^2 - y^2} $ 
        \item $f(x,y) = x^y$
    \end{itemize}
\end{multicols}

\bigskip

\textbf{Zadanie 6} Wykazać, że funkcja $f: \mathbb{R}^2 \to \mathbb{R}$

\[
f(x,y) =
\begin{cases}
    \frac{xy^2}{x^2 + y^2}, & \text{gdy~} (x,y) \neq (0,0) \\
    0, & \text{gdy~} (x,y) = (0,0)
\end{cases}
\] 

jest ciągła. Wyznaczyć jej pochodne cząstkowe funkcji $f$ w każdym punkcie
$(x, y) \in \mathbb{R}^2$ i pochodne kierunkowe w punkcie $(0, 0)$. Zbadać
różniczkowalność funkcji $f$ w każdym punkcie jej dziedziny.

\bigskip

\textbf{Zadanie 7} Zbadać różniczkowalność funkcji $f: \mathbb{R}^2 \to
\mathbb{R}$ oraz $g: \mathbb{R}^2 \to \mathbb{R}$ określonych wzorami

\[
f(x,y) = 
\begin{cases}
    \frac{x^3+y^3}{x^2+y^2}, & \text{gdy~} (x,y) \neq (0,0) \\
    0, & \text{gdy~} (x,y) = (0,0)
\end{cases}
\quad
g(x,y) =
\begin{cases}
    \frac{x^4+y^4}{x^2+y^2}, & \text{gdy~} (x,y) \neq (0,0) \\
    0, & \text{gdy~} (x,y) = (0,0)
\end{cases}
.\] 

\bigskip

\textbf{Zadanie 8} Nich $f: \mathbb{R} \times [0, 2\pi] \to \mathbb{R}^2$
będzie określona wzorem 

\[
f(r,\phi) = \left( r \cos\phi, r\sin \phi \right) 
.\] 

Znajdź macierz Jakobiego tego przekształcenia oraz oblicz jakobian
(wyznacznik macierzy jakobiego)

\bigskip

\textbf{Zadanie 9} Pole trapezu o podstawach $a$ oraz  $b$ i wysokości
$h$ jest dane wzorem {$S(a,b,h) = \frac{a+b}{2} h$}. Oblicz
$\frac{\partial S}{\partial a}, \frac{\partial S}{\partial b},
\frac{\partial S}{\partial h}$ i używając rysunku pokaż ich geometryczną
interpretacje.

\bigskip

\textbf{Zadanie 10} Oblicz $1.02^{3.01}$

\end{document}

