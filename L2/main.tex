%! TeX program = lualatex
%! TEX options = -synctex=1 -interaction=nonstopmode -file-line-error --shell-escape "%DOC%"
\documentclass[a4paper,11pt]{article}
\usepackage{polski}
% \usepackage{libertine}
\usepackage[top=1.5cm, bottom=1.5cm, left=2.5cm, right=2.5cm]{geometry}
\usepackage{url}
\usepackage{graphicx}
\usepackage{amsfonts}
\usepackage{amsmath}
\usepackage{enumitem}

\setlength\parindent{0.5pt} 

\begin{document}

\begin{center}
  {\large\textbf{Lista 2}}
\end{center}

\bigskip

\hrule

\begin{center}
    Zadania do deklaracji
\end{center}

\textbf{Zadanie 1} Korzystając ze wzoru Taylora zbadać zbieżność szeregu

\[
\sum_{n=1}^\infty \left(e^{1/\sqrt{n}} - 1 - \frac{1}{\sqrt{n}} 
- \frac{1}{2n}\right)
\]

\bigskip

\textbf{Zadanie 2} 
Z powodu swojej masy, przewody trakcyjne mają kształt opiwyswany przez funckję
${f(x) = a \cosh \frac{x}{a}}$, gdzie 
$\cosh x := \frac{e^x + e^{-x}}{2}$ 
to cosinus hiperboliczny (więcej informacji w internecie). Pokaż, że dla małych $x$ kształt jest dobrze przybliżony przez parabole
\[
f(x) = a + \frac{x^2}{2a}.
\]
Oszacuj bład dla dowolnego $a$ oraz $|x| \leq a$.

\bigskip

\textbf{Zadanie 3} 

a) Zapisać funkcję $f(x) = \frac{x}{1-x^2}$
jako sumę szeregu potęgowego wokół zera i wyznaczyć
przedział zbieżności tego szeregu.

b) Wyznaczyć przedział zbieżności szeregu potęgowego 
$\sum_{n=0}^\infty 2^{-n} x^{n^2}$.

\bigskip

\textbf{Zadanie 4} Znaleźć sumę szeregów
\begin{enumerate}[label=(\alph*)]
    \item $\displaystyle \sum \limits_{n=0}^\infty \frac{x^{2n+1}}{2n+1}$,
    \item $\displaystyle \sum \limits_{n=0}^\infty (-1)^{n} \frac{x^{2n+1}}{2n+1}$,
    \item $\displaystyle \sum \limits_{n=0}^\infty \frac{x^{2n}}{2n!}$.
\end{enumerate}

\hrule

\begin{center}
    Zadanie pozostałe
\end{center}

\textbf{Zadanie 5} Korzystając ze wzoru Taylora zbadać zbieżność szeregu

\[
\sum_{n=0}^\infty \left( \ln(1+n) - \ln(n) - \frac{1}{n}\right)
\]

Wskazówka: nie rozwijać Taylora w zerze tylko w trochę innym punkcie. \\
Wskazówka 2: zobaczyć co zwraca WolframAlpha na zapytanie $\ln(1+x)$

\bigskip

\textbf{Zadanie 6} Zbadać wypukłość i znaleźć punkty przegięcia funkcji:

\begin{enumerate}[label=(\alph*)]
    \item $f(x) = x^2\ln x \text{~~dla~~} x > 0$
    \item $f(x) = x - \sin(x)$
    \item $f(x) = (1+x^2)e^x$
\end{enumerate}

\bigskip

\textbf{Zadanie 7} Niech

\[
f(x) = 
\begin{cases}
    \frac{e^x-1}{x} & \text{~gdy~} x\neq0, \\
    1 & \text{~gdy~} x=0
\end{cases}
\]
Wykazać, że funkcja $f$ jest klasy $C^\infty$
i znaleźć wartości jej wszystkich pochodnych w zerze.

\bigskip

\textbf{Zadanie 8} 
\begin{enumerate}[label=(\alph*)]
    \item Znaleźć promień zbieżności szeregu potęgowego
    \[
    \sum_{n=1}^\infty 2^{5^n} x^{a_n},
    \]
    gdzie $a_1 = 1$ oraz $a_{n+1} = 5a_n + (-3)^n$ dla $n\geq1$.
    %
    \item Rozwinąć w szereg potęgowy wokół punktu $x_0 = 0$ funkcję 
    $f:\mathbb{R} \setminus \{3/2\} \to \mathbb{R}$
    \[
    f(x) = \arctan \left( \frac{3x+8}{4x-6} \right)
    \]
    Wyznaczyć przedział zbieżności otrzymanego szeregu. 
    Znaleźć wartość sumy szeregu dla $x = 7/4$.
\end{enumerate}


\end{document}

