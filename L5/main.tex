%! TeX program = lualatex
%! TEX options = -synctex=1 -interaction=nonstopmode -file-line-error --shell-escape "%DOC%"
\documentclass[a4paper,11pt]{article}
\usepackage{polski}
% \usepackage{libertine}
\usepackage[top=1.5cm, bottom=1.5cm, left=2.5cm, right=2.5cm]{geometry}
\usepackage{url}
\usepackage{graphicx}
\usepackage{amsfonts}
\usepackage{amsmath}
\usepackage{enumitem}
\usepackage{multicol}
\usepackage{fontspec}

\newfontfamily\DejaSans{DejaVu Sans}

\everymath{\displaystyle}

\setlength\parindent{0.5pt} 

\begin{document}

\begin{center}
  {\large\textbf{Lista 5}}
\end{center}

\hrulefill
\begin{center}
    \textbf{Zadania do deklaracji (poniedziałek)}
\end{center}

\bigskip

\textbf{Zadanie 1} Oblicz następujące całki nieoznaczone:

\begin{multicols}{3}
    \begin{enumerate}
        \item $\int \frac{\arcsin x}{x^2} dx$
        \item $\int \frac{\sin^2 x}{e^x}dx$
        \item $\int \cos^2(\ln x)dx$
    \end{enumerate}
\end{multicols}

\bigskip

\textbf{Zadanie 2} Oblicz następujące całki oznaczone:

\begin{multicols}{3}
    \begin{enumerate}
        \item $\int \limits_0^1 x(1-x)^{42} dx$
        \item $\int \limits_0^{\frac{\pi}{4}} \frac{\sin x}{1+\cos x}dx$
        \item $\int \limits_{-\frac{\pi}{2}}^{\frac{\pi}{2}} 
            \frac{1+(\sin x)^{2023}}{1+\cos x} dx$
    \end{enumerate}
\end{multicols}

\bigskip

\textbf{Zadanie 3} Obliczyć granice sprowadzając je do granic
odpowiednich sum Riemanna: (wystarczy zrobić tylko podpunkt 1. żeby móc
zadeklarować zadanie)

\begin{multicols}{2}
    \begin{enumerate}
        \item $ \lim_{n\to \infty} \sum_{k=1}^n \frac{k+n}{3k^2 + n^2}$
        \item $\lim_{n \to  \infty} \frac{\sqrt[n]{(n+1)(n+2)\ldots(n+n)}
            }{n}$
    \end{enumerate}
\end{multicols}

Wskazówka: popatrzeć na przykład 9.36 w skrypcie prowadzącego.
\bigskip

\textbf{Zadanie 4} Obliczyć całki niewłaściwe:

\begin{multicols}{3}
    \begin{enumerate}
        \item $ \int_{0}^\infty \frac{dx}{x(\ln^2 x + 1)}$
        \item $\int_{-1}^{0} \frac{e^{\frac{1}{x}}}{x^3} dx$
        \item $\int_{0}^{1} \sin(\ln x) dx$
    \end{enumerate}
\end{multicols}

\bigskip

\hrulefill

\begin{center}
    \textbf{Zadania na zajęcia}
\end{center}

\bigskip

\textbf{Zadanie 5} Wykaż tożsamość

$$\int_{0}^{1} \frac{1}{x^x} dx = \sum_{n=1}^{\infty} \frac{1}{n^n}$$

Wskazówka: użyj triku $x = e^{\ln x}$ oraz przedstaw $e^x$ w postaci
szeregu nieskończonego

\bigskip

\textbf{Zadanie 6$^{\ast\ast}$} Zdefiniujmy abstrakcyjny iloczyn skalarny
w przestrzeni wszystkich funkcji całkowalnych z kwadratem (czyli takich
funkcji $f$ dla których zachodzi $\int_{-1}^1 f^2(x) dx < \infty$ ), jako 

$$ \langle f,g\rangle = \int_{-1}^1 f(x) g(x) dx.$$

Pokaż, że wielomany Legandra $P_n(x)$, zdefinowane jako takie wielomiany
które spełniają równanie

$$[(1-x^2)P_n'(x)]' + n(n+1) P_n(x) = 0, \quad n \in \mathbb{Z}_0^{+}$$

są ortogonalne, tzn. 

$$\langle P_m, P_n \rangle = 0, \quad \text{dla~} m \neq n$$

\bigskip

\textbf{Zadanie 7} Pokaż, że 
\begin{itemize}
    \item $\int^\infty_{-\infty} e^{-x^2} dx = 2 \int_{0}^\infty
        e^{-x^2} dx = \int_{0}^\infty \frac{e^{-x}}{\sqrt{x} }$
    \item $\int_{0}^1 \frac{dx}{\arccos x} = \int_{0}^{\frac{\pi}{2}}
        \frac{\sin x}{x} dx$
    \item $\int_{0}^{\frac{\pi}{2}} f(\sin x) dx =
        \int_{0}^{\frac{\pi}{2}} f(\cos x) dx$
\end{itemize}

\bigskip

\textbf{Zadanie 8} {\DejaSans ☺} \textit{(Proszę nie brać tego zadania na
serio)} Pokaż, że 

\begin{itemize}
    \item $\sum_{n=0}^\infty (-1)^n = 1 -1 + 1 - 1 + \ldots = \frac{1}{2}$
    \item $\sum_{n=0}^\infty (-1)^{n-1} n = 1 -2 + 3 - 4 + \ldots =
        \frac{1}{4}$
\end{itemize}

\textit{Wskazówka: przedstawić te szeregi jako szeregi potęgowe w punkcie
$x=1$ i skorzystać z Tw. Abela, nie zważając na jego założenia}

\bigskip

\textbf{Zadanie 9} Ile cyfr w zapisie dziesiętnym ma liczba $100!$?

\bigskip

\textbf{Zadanie 10} Udowodnić, że jeśli funkcja $f:[0,2\pi] \to
\mathbb{R}$ jest ciągła to 

$$\lim_{n\to \infty} \int_{0}^{2\pi} f(x) \sin(nx) dx = 0.$$

\bigskip

\textbf{Zadanie 11$^\ast$} Niech $f: [a,b] \to  [a',b']$ będzie dodatnią,
rosnącą i ciągłą bijekcją. Wykazać, że 

$$\int_{a}^b f(x) dx + \int_{a'}^{b'} f^{-1}(y) dy = bb' - aa'.$$

\textit{Wskazówka:} Wykorzystać interpretację geometryczną całki.


\hrulefill

\bigskip

\textbf{Zadanie ciekawostka} Objętość kuli o promieniu $1$ w $2n$
wymiarach wynosi $V_{2n} = \frac{\pi^{n}}{n!}$. Pokaż, że $\lim_{n\to
\infty} V_{2n} = 0$

\end{document}

