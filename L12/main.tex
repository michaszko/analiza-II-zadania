%! TeX program = lualatex
%! TEX options = -synctex=1 -interaction=nonstopmode -file-line-error --shell-escape "%DOC%"
\documentclass[a4paper,11pt]{article}
\usepackage{polski}
% \usepackage{libertine}
\usepackage[top=1.5cm, bottom=1.5cm, left=2.5cm, right=2.5cm]{geometry}
\usepackage{url}
\usepackage{graphicx}
\usepackage{amsfonts}
\usepackage{amsmath}
\usepackage{enumitem}
\usepackage{multicol}
\usepackage{fontspec}
\usepackage{bm}

\newfontfamily\DejaSans{DejaVu Sans}

\everymath{\displaystyle}
\newcommand{\dm}[1]{\displaystyle{#1}}
\newcommand{\RR}{\mathbb{R}}
\newcommand{\CC}{\mathbb{C}}

\setlength\parindent{0.5pt} 

\begin{document}

\begin{center}
  {\large\textbf{Lista 12}}
\end{center}

\hrulefill
\begin{center}
    \textbf{Zadania do deklaracji (poniedziałek)}
\end{center}

\bigskip

\textbf{Zadanie 1} 

\bigskip

\textbf{Zadanie 2} 

\bigskip

\textbf{Zadanie 3} 

\bigskip

\textbf{Zadanie 4} 

\bigskip

\hrulefill

\bigskip

\textbf{Zadanie 5} Podaj przykład zbioru $A \subset \RR$ o mierze
$\lambda_1(A) = 1$, takiego, że  $\int_A x^2 dx = +\infty$.

\bigskip

\textbf{Zadanie 6} Oblicz:

\begin{enumerate}
  \item $\int_T \cos(x+y) dxdy$, gdzie $T$- pełen trójkąt ograniczony
    prostymi o równaniach:  $x=0, y=\pi, x=y$;
  \item  $\int_{K(0,1)} xy dx dy$, gdzie  $K(0,1)$ to koło o środku w
    $(0,0)$ oraz promieniu 1;
  \item $\int_{K_{+1}} z dxdydz$, gdzie $K_{+-} = \{\bm{x} \in \RR^3:
    x>0, y<0, \|\bm{x}\| <1\} $
  \item $\int_{P_{1,2}} (x^2 + y^2) dxdy$, gdzie $P_{1,2}$ -- pierścień
    kołowy („pełny”) na płaszczyźnie, o środku
    0 i promieniu 1 (wewnętrznym) oraz 2 (zewnętrznym)
  \item objętość walca obrotowego o wysokości h i promieniu podstawy r,
  \item  $^\ast$ Objętość „pełnego” torusa powstałego przez obrót koła w
    płaszczyźnie „$x, z$” o
    środku $(R, 0, 0)$ i promieniu $r$ wokół osi $z$ $(0 < r < R)$
  \item $\int_{D_\alpha} (x-y) dxdy$ gdzie $D_\alpha = \{\bm{x} \in
      \RR^2: x\ge 1, x - \frac{1}{x^\alpha} \le y \le x +
    \frac{1}{x^\alpha}\}$ dla $\alpha > 0$
  \item pole powierzchni elipsy o równaniu $\frac{x^2}{a^2} +
    \frac{y^2}{b^2} = 1$
  \item $\int_{R} (x-y) e^{x^2 - y^2} dxdy$, gdzie $R$ to pole
    ograniczone przez krzywe  $x+y=1, x+y=3, x^2-y^2=-1, x^2-y^2=1$
\end{enumerate}

\end{document}

